\documentclass[A4,11pt]{article}
\usepackage{german}

\begin{document}
\title{CS102\quad\LaTeX\quad\"Ubung}
\maketitle
\date
\begin{center}
{Yanik Weber}
\end{center}
\section{Das ist der erste Abschnitt}
hier k\"onnte auch etwas anderes stehen
\section{Tabelle}
Unsere wichtigsten Daten finden Sie in Tabelle 1.
\begin{table}[h]
\begin{tabular}{c|c|c|c}
&Punkte erhalten& Punkte m\"oglich & \% \\
\hline
Aufgabe 1& 2& 4& 0.5\\

Aufgabe 2& 7& 8& 10\\

Aufgabe 3& 3& 3& 1\\
\end{tabular}
\caption{Diese Tabelle könnte auch andere Werte anzeigen}
\end{table}
\section{Formeln}
\subsection{Pythagoras}
Der Satz des Pythagoras errechnet sich wie folgt: $ a^{2}+b^{2}=c^{2} $ Daraus k\"onnen wir die Lösungen wie folgt berechnen: $ c= \sqrt{a^{2}+b^{2}} $
\subsection{Summen}
Wir k\"onnen auch die Formel f\"ur eine Summe angeben:\\
\begin{center}
$ s=\sum_{i=1}^n i= \frac{n\ast(n+1)}{2} $
\end{center} 
\end{document}
